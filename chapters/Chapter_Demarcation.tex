%************************************************
\chapter{Abgrenzung}\label{ch:demarcation}
%************************************************

%\section{Annahmen}
%\todo{Annahmen schreiben}
%\begin{itemize}
%	\item Grundlegendes Verständis wie das Internet Funktioniert
%	\item js html css
%\end{itemize}
%\section{Abgrenzung}
Im Rahmen dieser Arbeit werden ausschließlich \pTp-basierte Ansätze für die Inhaltsverteilung untersucht. Es wird untersucht, ob ein rein browserbasiertes \cdn geeignet ist, die benötigte Bandbreite von Live-Streams beim Einsatz in Unternehmen und in Schulen zu reduzieren. Nicht näher betrachtet werden die Verteilung von On-Demand-Videos sowie Lösungsansätze, die eine Installation weiterer Software benötigen. Da es in dieser Arbeit um die Reduzierung von Bandbreite in geteilten Netzwerken geht, wird nicht näher untersucht, ob das \cdn ebenfalls dazu geeignet ist, Inhalte über ein lokales Netzwerk hinaus zu verteilen. Auch wenn das vorgestellte \cdn sowohl in der Schul-Cloud als auch bei Slidesync produktiv eingesetzt werden soll, ist es nicht Ziel dieser Arbeit, alle dafür notwendigen Hintergründe zu behandeln. Wichtige dafür notwendige Themenbereiche, wie z.B. DRM-Lizensierung, werden nicht näher betrachtet und bedürfen weiterer Untersuchung. Vielmehr handelt es sich um eine Machbarkeitsstudie, die ermitteln soll, ob der Ansatz in der Praxis eingesetzt werden kann.

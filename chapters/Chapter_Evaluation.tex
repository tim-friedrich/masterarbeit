\chapter{Evaluation}\label{ch:evaluation}

\section{Prequisites}

\section{Techincal evaluation}

\subsection{Bandwidth}

\subsubsection{Simulierter Workload}

\subsubsection{Educational context}

\subsubsection{Live streaming in the coporate context}

\subsection{Nutzerzufriedenheit}
wie soll das aussehen?

\section{Browser compatbility}

\subsection{Browser Usage in coropate networks}

\subsection{Browser usage in eduational networks}

\section{Security considerations}
% Wikipedia: (https://de.wikipedia.org/wiki/WebRTC)
%P-Leak
%In Verbindung mit WebRTC können private IP-Adressen trotz VPN-Verbindung über JavaScript ausgelesen werden.[19] Das Beispiel Firefox Hello schließt Rechner hinter einer Firewall und mit privaten IP-Adressen aus. Deshalb kann eine Website mit JavaScript einen STUN-Server nach der tatsächlichen IP-Adresse fragen lassen. Dies hat zur Folge, dass Anonymisierungsdienste ihren Zweck nicht mehr erfüllen und keinen Schutz mehr vor einem IP-Leak bieten können.
%
%Gegenmaßnahmen
%Zum Schutz vor einem IP-Leak bieten sich zwei Vorgehensweisen an. Eine Option bietet die Installation von Add-Ons/Plugins zur Verhinderung der Weitergabe der öffentlichen IP-Adresse, beispielsweise "WebRTC Leak Prevent"[20] oder "Easy WebRTC Block".[21][22]
%
%Die andere Möglichkeit ist eine Änderung der Einstellungen im Browser. Im Firefox kann über about:config der Wert media.peerconnection.enabled auf false gesetzt werden, wodurch ein IP-Leak verhindert wird.[23]
%
%Die gemeinhin als Werbeblocker bekannte plattformübergreifende Browser-Erweiterung uBlock Origin zum Filtern von Webinhalten bietet in ihren Einstellungen im Abschnitt "Privatsphäre" die zuschaltbare Möglichkeit, die Freigabe der lokalen IP-Adresse via WebRTC zu verhindern.

\section{DRM licencing}
%passt hier nicht wirklich hin
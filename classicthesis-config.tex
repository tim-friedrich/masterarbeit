
\hyphenation{brow-ser}
\hyphenation{data-base}
\hyphenation{ent-spre-chen-den}
\hyphenation{da-durch}

% \hyphenation{Gül-tig-keit}

% ********************************************************************
% Set the encoding of your files.
% ********************************************************************

\PassOptionsToPackage{utf8}{inputenc}	% latin9 (ISO-8859-9) = latin1+"Euro sign"
 \usepackage{inputenc}

% ********************************************************************
% Configure classicthesis for your needs here, e.g., remove "drafting" below
% ********************************************************************

\PassOptionsToPackage{eulerchapternumbers,
						listings,
						drafting,
					 	pdfspacing,
					 	linedheaders,
					 	subfig,
					 	beramono,
					 	eulermath,
					 	parts,
					 	dottedtoc
					 }{classicthesis}

% ********************************************************************
% Available options for classicthesis.sty
% (see ClassicThesis.pdf for more information):
% drafting
% parts nochapters linedheaders
% eulerchapternumbers beramono eulermath pdfspacing minionprospacing
% tocaligned dottedtoc manychapters
% listings floatperchapter subfig
% ********************************************************************


% ********************************************************************
% Personal data and user ad-hoc commands
% ********************************************************************
\newcommand{\myTitle}{Distribution of large data in networks with limited bandwidth\xspace}
\newcommand{\myGermanTitle}{Webbasierte Verteilung großer Datenmengen in lokalen Netzwerken
\xspace}
\newcommand{\myDegree}{Bachelor of Science\xspace}
\newcommand{\myName}{Tim Friedrich\xspace}
\newcommand{\myEmail}{tim.friedrich@student.hpi.de\xspace}
\newcommand{\myProf}{Prof. Dr. Christoph Meinel\xspace}
\newcommand{\mySupervisor}{Dipl.-Inf. (FH). Jan Renz\xspace}
\newcommand{\myFaculty}{Hasso Plattner Institute\xspace}
\newcommand{\myDepartment}{Internet Technologies and Systems\xspace}
\newcommand{\myUni}{University of Potsdam\xspace}
\newcommand{\myLocation}{Potsdam\xspace}
\newcommand{\myTime}{October 5\textsuperscript{th}, 2016\xspace}
\newcommand{\cdn}{CDN\xspace}
\newcommand{\cdns}{CDNs\xspace}
\newcommand{\pTp}{Peer To Peer\xspace}
\newcommand{\webrtc}{Webrtc\xspace}
\newcommand{\schulCloud}{Schul-Cloud\xspace}
\newcommand{\hpi}{Hasso-Plattner-Instituts\xspace}
\newcommand{\study}{\textsc{\emph{\color{red} Studie }}\xspace}
\newcommand{\note}[1]{\emph{\color{red} #1 }}



% ********************************************************************
% Setup, finetuning, and useful commands
% ********************************************************************

\newcounter{dummy} % necessary for correct hyperlinks (to index, bib, etc.)
\newlength{\abcd} % for ab..z string length calculation
\providecommand{\mLyX}{L\kern-.1667em\lower.25em\hbox{Y}\kern-.125emX\@}
\newcommand{\ie}{i.\,e.}
\newcommand{\Ie}{I.\,e.}
\newcommand{\eg}{e.\,g.}
\newcommand{\Eg}{E.\,g.}

% ********************************************************************
% Packages with options that might require adjustments
% ********************************************************************

\PassOptionsToPackage{ngerman}{babel}   % change this to your language(s)
% Spanish languages need extra options in order to work with this template
%\PassOptionsToPackage{spanish,es-lcroman}{babel}
	\usepackage{babel}

\usepackage{csquotes}
\PassOptionsToPackage{%
    %backend=biber, %instead of bibtex
	backend=bibtex8,bibencoding=ascii,%
	language=auto,%
	style=numeric-comp,%
    %style=authoryear-comp, % Author 1999, 2010
    %bibstyle=authoryear,dashed=false, % dashed: substitute rep. author with ---
    sorting=nyt, % name, year, title
    maxbibnames=10, % default: 3, et al.
    %backref=true,%
    natbib=true % natbib compatibility mode (\citep and \citet still work)
}{biblatex}
    \usepackage{biblatex}

%bugfix
\makeatletter
\def\blx@maxline{77}
\makeatother

\PassOptionsToPackage{fleqn}{amsmath}       % math environments and more by the AMS
    \usepackage{amsmath}

% ********************************************************************
% General useful packages
% ********************************************************************

\PassOptionsToPackage{T1}{fontenc} % T2A for cyrillics
    \usepackage{fontenc}
\usepackage{textcomp} % fix warning with missing font shapes
%\usepackage{scrhack} % fix warnings when using KOMA with listings package
\usepackage{xspace} % to get the spacing after macros right
\usepackage{mparhack} % get marginpar right
\PassOptionsToPackage{printonlyused
						%,smaller
						}{acronym}
    \usepackage{acronym} % nice macros for handling all acronyms in the thesis
    %\renewcommand{\bflabel}[1]{{#1}\hfill} % fix the list of acronyms --> no longer working
    %\renewcommand*{\acsfont}[1]{\textsc{#1}}
    %\renewcommand*{\aclabelfont}[1]{\acsfont{#1}}

% ********************************************************************
% Setup floats: tables, (sub)figures, and captions
% ********************************************************************

\usepackage{tabularx} % better tables
    \setlength{\extrarowheight}{3pt} % increase table row height
\newcommand{\tableheadline}[1]{\multicolumn{1}{c}{\spacedlowsmallcaps{#1}}}
\newcommand{\myfloatalign}{\centering} % to be used with each float for alignment
\usepackage{caption}
% Thanks to cgnieder and Claus Lahiri
% http://tex.stackexchange.com/questions/69349/spacedlowsmallcaps-in-caption-label
% [REMOVED DUE TO OTHER PROBLEMS, SEE ISSUE #82]
%\DeclareCaptionLabelFormat{smallcaps}{\bothIfFirst{#1}{~}\MakeTextLowercase{\textsc{#2}}}
%\captionsetup{font=small,labelformat=smallcaps} % format=hang,
\captionsetup{font=small} % format=hang,
\usepackage{subfig}

% ********************************************************************
% Using PDFLaTeX
% ********************************************************************
\PassOptionsToPackage{pdftex,hyperfootnotes=true,pdfpagelabels}{hyperref}
    \usepackage{hyperref}  % backref linktocpage pagebackref
\pdfcompresslevel=9
\pdfadjustspacing=1
\PassOptionsToPackage{pdftex}{graphicx}
\usepackage{graphicx}


% ********************************************************************
% Hyperreferences
% ********************************************************************
\hypersetup{%
    %draft, % = no hyperlinking at all (useful in b/w printouts)
    %colorlinks=true, linktocpage=true, pdfstartpage=3, pdfstartview=FitV,%
    % uncomment the following line if you want to have black links (e.g., for printing)
    colorlinks=false, linktocpage=false, pdfstartpage=3, pdfstartview=FitV, pdfborder={0 0 0},%
    breaklinks=true, pdfpagemode=UseNone, pageanchor=true, pdfpagemode=UseOutlines,%
    plainpages=false, bookmarksnumbered, bookmarksopen=true, bookmarksopenlevel=1,%
    hypertexnames=true, pdfhighlight=/O,%nesting=true,%frenchlinks,%
    urlcolor=webbrown, linkcolor=RoyalBlue, citecolor=webgreen, %pagecolor=RoyalBlue,%
    %urlcolor=Black, linkcolor=Black, citecolor=Black, %pagecolor=Black,%
    pdftitle={\myTitle},%
    pdfauthor={\textcopyright\ \myName, \myUni, \myFaculty},%
    pdfsubject={},%
    pdfkeywords={},%
    pdfcreator={pdfLaTeX},%
    pdfproducer={LaTeX with hyperref and classicthesis}%
}

% ********************************************************************
% Setup autoreferences
% ********************************************************************
% There are some issues regarding autorefnames
% http://www.ureader.de/msg/136221647.aspx
% http://www.tex.ac.uk/cgi-bin/texfaq2html?label=latexwords
% you have to redefine the makros for the
% language you use, e.g., american, ngerman
% (as chosen when loading babel/AtBeginDocument)
% ********************************************************************
\makeatletter
\@ifpackageloaded{babel}%
    {%
       \addto\extrasamerican{%
			\renewcommand*{\figureautorefname}{Figure}%
			\renewcommand*{\tableautorefname}{Table}%
			\renewcommand*{\partautorefname}{Part}%
			\renewcommand*{\chapterautorefname}{Chapter}%
			\renewcommand*{\sectionautorefname}{Section}%
			\renewcommand*{\subsectionautorefname}{Section}%
			\renewcommand*{\subsubsectionautorefname}{Section}%
                }%
       \addto\extrasngerman{%
			\renewcommand*{\paragraphautorefname}{Absatz}%
			\renewcommand*{\subparagraphautorefname}{Unterabsatz}%
			\renewcommand*{\footnoteautorefname}{Fu\"snote}%
			\renewcommand*{\FancyVerbLineautorefname}{Zeile}%
			\renewcommand*{\theoremautorefname}{Theorem}%
			\renewcommand*{\appendixautorefname}{Anhang}%
			\renewcommand*{\equationautorefname}{Gleichung}%
			\renewcommand*{\itemautorefname}{Punkt}%
                }%
            % Fix to getting autorefs for subfigures right (thanks to Belinda Vogt for changing the definition)
            \providecommand{\subfigureautorefname}{\figureautorefname}%
    }{\relax}
\makeatother

% ********************************************************************

\usepackage{classicthesis}

% ********************************************************************
% Setup code listings
% ********************************************************************

\usepackage{minted}
\PassOptionsToPackage{newfloat}{minted}

\usepackage{scrhack}

\newcommand{\mintedstylefix}{lovelace}

\setminted{
	numbers=left,
	stepnumber=1,
	numbersep=8pt,
	frame=lines,
	framesep=2mm,
	breaklines=true,
	breakafter={:-/.},
	breakaftersymbolpre={},
	linenos
}

\setmintedinline{
	breaklines=true,
	breakafter={-_/.},
	breakaftersymbolpre={}
}

% ********************************************************************

\providecommand*{\listingautorefname}{Listing}

%\renewcommand{\PrelimText}{\hspace{2.1cm} \footnotesize\texttt{[Draft -- \today\ at \thistime\\]}}

\usepackage{etoolbox}

\newcommand\todo[2][]{%
  \ifstrempty{#1}{%
    {\marginpar{\color{orange}{#2}}}
  }{%
    {\color{orange}{#1}}{\marginpar{\color{orange}{#2}}}
  }%
}

\renewcommand{\texttt}[1]{%
  \begingroup
  \ttfamily
  \begingroup\lccode`~=`/\lowercase{\endgroup\def~}{/\discretionary{}{}{}}%
  \begingroup\lccode`~=`[\lowercase{\endgroup\def~}{[\discretionary{}{}{}}%
  \begingroup\lccode`~=`.\lowercase{\endgroup\def~}{.\discretionary{}{}{}}%
  \catcode`/=\active\catcode`[=\active\catcode`.=\active
  \scantokens{#1\noexpand}%
  \endgroup
}

\newcommand\textvtt[1]{{\normalfont\fontfamily{cmvtt}\selectfont #1}}

\usepackage{float}

\usepackage{longtable}
\usepackage{booktabs}
\usepackage{multirow}

\usepackage{enumitem}

\usepackage{cleveref}
\PassOptionsToPackage{capitalise,noabbrev}{cleveref}

\usepackage{arydshln}
